%% Template for a preprint Letter or Article for submission
%% to the journal Nature.
%% Written by Peter Czoschke, 26 February 2004
%%

\documentclass{naturemod}

\usepackage{geometry} 
\usepackage{graphicx}
\usepackage{amssymb}
%\usepackage{epstopdf}
\usepackage{booktabs}
%\usepackage{natbib}
\usepackage{amsmath,bm}

% style for partial derivative
\newcommand{\pd}[2]{ \frac{\partial #1}{\partial #2} }
%\newcommand{\pd}[2]{ \partial_{#2}#1 }
\newcommand{\od}[2]{\ensuremath{\frac{d #1}{d #2}}}
\newcommand{\td}[2]{\ensuremath{\frac{D #1}{D #2}}}
\newcommand{\ab}[1]{\ensuremath{\langle #1 \rangle}}
%\newcommand{\dab}[1]{\ensuremath{\llangle #1 \rrangle}}
\newcommand{\bss}[1]{\textsf{\textbf{#1}}}
\newcommand{\ol}{\ensuremath{\overline}}
\newcommand{\olx}[1]{\ensuremath{\overline{#1}^x}}
\newcommand{\nms}{\ensuremath{\mbox{ N m}^{-2}}}
\newcommand{\wms}{\ensuremath{\mbox{ W m}^{-2}}}
\newcommand{\mms}{\ensuremath{\mbox{ m}^2 \mbox{ s}^{-1}}}
\newcommand{\mss}{\ensuremath{\mbox{ m s}^{-2}}}
\newcommand{\ihat}{\hat{\textbf{\i}}}
\newcommand{\jhat}{\hat{\textbf{\j}}}
\newcommand{\orho}{\frac{1}{\rho_0}}


%% make sure you have the nature.cls and naturemag.bst files where
%% LaTeX can find them

\bibliographystyle{naturemag}

\title{Antarctic Sea Ice Drift Sustains Global Overturning Circulation}

%% Notice placement of commas and superscripts and use of &
%% in the author list

\author{Ryan Abernathey$^2$, Paul Holland$^2$ , Emily Newsom$^3$,
		Matt Mazloff$^4$, Ivana Cerovecki$^4$, \& Lynne Talley$^5$}


\begin{document}

\maketitle

\begin{affiliations}
 \item Lamont Doherty Earth Observatory of Columbia University
 \item British Antarctic Survey
 \item University of Washington
 \item Scripps Institution of Oceanography
\end{affiliations}

\begin{abstract}
The ocean's overturning circulation transports water in and out of the otherwise isolated deep ocean. Because the deep ocean is capable of storing large quantities of heat and carbon, the overturning circulation has been shown to play a central role in long-term climate fluctuations such as ice ages and will also regulate the future evolution of Earth's climate under anthropogenic change\cite{SarmientoToggweiler1984,SigmanBoyle2000,ToggweilerRussell2008}. An important component in the global overturning ``conveyor belt'' is Southern Ocean upwelling; the steeply sloping density surfaces of the Antarctic Circumpolar Current allow circumpolar deep water to travel adiabatically from the interior to the surface, where its density can be modified through surface fluxes of heat and freshwater\cite{ToggweilerSamuels1995,WolfeCessi2011,NikurashinVallis2012}. Here we show that freshwater fluxes produced by freezing and melting sea ice account for a large fraction of the density transformation of upwelling circumpolar deep water, and that these fluxes depend heavily on sea-ice drift away from Antarctica. This suggests a central role for Antarctic sea ice in the ocean overturning circulation and global climate system.
\end{abstract}

% papers about salinity
% ToggweilerSamuels1995b - Effect of Sea Ice on the Salinity of Antarctic Bottom Waters - talks about the competing effects of brine rejections and precip / runoff. Very relevant, with good citations.  
% ToggweilerSamuels1995 - Effect of Drake Passage on the global thermohaline circulation
% Gill1973 - Circulation and bottom water production in the Weddell Sea


%The ocean is stratified in density, and as water circulates between the surface and the deep, its density must be modified  (transformed) thermodynamically. The buoyancy of seawater is primarily a function of its temperature and salinity. Transformation can take the form of air-sea fluxes of heat and freshwater, which generally work to create density contrasts, and mixing in the ocean interior, which homogenizes density.

It has long been known that deep, dense waters are formed in the high-latitude North Atlantic and Southern Ocean\cite{Broecker1991}. These abyssal water masses, known as {\em North Atlantic Deep water} (NADW) and {\em Antarctic Bottom Water} (AABW) represent most of the ocean volume. A more difficult question is how deep waters regain buoyancy. Originally, it was hypothesized that mixing with lighter waters above was the primary mechanism for ``closing'' the overturning circulation\cite{StommelArons1959,Munk1966,Ferrari2014}. While this pathway is certainly essential, more recent attention has emphasized how the steeply sloping isopycnals of the Southern Ocean, a consequence of the strong westerly winds and the zonally unblocked geometry of the basin, provide a pathway for deep water to upwell adiabatically, i.e. without mixing\cite{ToggweilerSamuels1995,WolfeCessi2011,NikurashinVallis2012}. Once near the surface, the upwelled {\em Circumpolar Deep Water} (CDW) can be modified directly by surface fluxes. For lighter "upper" CDW, hydrographic evidence suggests transformation to lighter waters and recirculation towards the low-latitude oceans as a combination of mode, intermediate, and surface water\cite{MarshallSpeer2012,RintoulNaveiraGarabato2013}. But for denser, "lower" CDW (including most NADW), observations indicate transformation to denser water and possibly recirculation within the AABW cell\cite{Talley2013}. This ``divergence'' of upwelling CDW is central feature of the global overturning circulation and has major consequences for the ocean carbon cycle\cite{MarinovEtAl2006}. The results described below show how freshwater fluxes related to Antarctic sea ice drift help sustain this divergence.

The importance of sea ice freshwater redistribution for the production of AABW was recognized early on\cite{Gill1973}, but observational limitations have prevented a global-scale estimate of the magnitude of this effect. In particular, sea ice thickness measurements (both {\em in situ} and remotely sensed) do not provide sufficient spatial and temporal resolution to constrain the local freshwater exchange with the ocean\cite{RenEtAl2011,HollandKwok2012,HollandEtAl2014}. Our approach here is to use a data-assimilating numerical model, whose sea ice cycle has been validated against the available observations, to examine the ocean circulation and air-ice-ocean fluxes in an integrated way. We employ the Southern Ocean State Estimate (SOSE), a high resolution general circulation model that has been constrained, via an adjoint method, to nearly all available observations in the 2005-2011 period (see {\em Methods} for details). This approach provides an optimal synthesis of the observations that is also consistent with physics. Crucially, SOSE includes a dynamic sea ice model which shows strong agreement with observed sea ice thickness, concentration, and velocity data, meaning it can provide an accurate estimate of the freshwater budget (see supplementary material for model validation).

The role of sea-ice in the freshwater cycle can be assessed by comparing the freshwater flux leaving the atmosphere with that entering the ocean; the difference is due to sea ice freshwater redistribution. Redistribution encompasses two distinct processes: 1) freshwater is frozen out of seawater (i.e.~``brine rejection''), transported laterally, and then melted; 2) snow falls onto the ice, is transported laterally, and then melted. Fig.~1 shows the freshwater flux from the atmosphere, the effect of sea ice redistribution (ice motion vectors indicate the direction of transport), and the freshwater reaching the ocean. From the atmosphere, precipitation dominates over the Southern Ocean, leading to a broadly distributed positive freshwater flux with a characteristic magnitude of 0.5 m / year, and runoff provides a stronger freshwater source near the Antarctic coast. But sea ice redistribution reverses the sign of the freshwater flux near the coast, both through brine rejection and simply by intercepting falling snow. Freshwater is then transported away from Antarctica and melted near the polar front, where the melt flux equals or exceeds the precipitation. As a result, the net freshwater flux to the ocean closely resembles the redistribution flux, with much stronger gradients than the atmospheric fluxes could produce. Because of this redistribution, the magnitude of the annual mean freshwater flux both into and out of the Southern Ocean exceeds one meter per year, a very large value on the global scale\cite{PeixotoOort1992,SchanzeEtAl2010}.

\begin{figure}
\begin{center}
\includegraphics{../../figures_ice_moc/fw_flux_3.pdf}
\caption{Mean freshwater fluxes from SOSE for the period 2005-2011. The Left panel shows the flux leaving the atmosphere and land, the and right panel shows the flux entering the ocean. The center panel shows the sea ice redistribution flux, with vectors indicating the lateral sea ice thickness transport.}
\label{default}
\end{center}
\end{figure}

To quantify the impact of freshwater redistribution on ocean circulation, we calculate the surface water mass transformation due to heat and freshwater fluxes (see {\em Methods} for details). The transformation rate quantifies the rate at which a given water mass (here defined by its potential density), is made lighter (negative transformation rate) or denser (positive) due to diabatic processes. If the transformation rate is is convergent or divergent at a particular density class, water must be subducted to or upwelled from the interior in that class order to satisfy mass continuity\cite{MarshallEtAl1999,LargeNurser2001}. Water mass transformation thus provides a quantitative link between between surface fluxes and overturning circulation. The transformation rates in SOSE are plotted in Fig.~2a. Although interior mixing provides additional transformation below the surface, the overturning circulation can readily be deduced from the surface transformation rate; nearly 40 Sv of CDW upwell between densities $\sigma_\theta =  26.9$ and $\sigma_\theta = 27.5$. Water lighter than $\sigma_\theta=27.4$ is made less dense; this represents the so-called ``upper branch'' of the overturning circulation. Water denser than $\sigma_\theta=27.4$ is made denser, eventually becoming AABW. (As with most numerical models, the production of AABW is too weak in SOSE, probably due to poor representation of coastal polynyas.) It is noteworthy that heat fluxes play a secondary role in the transformation of CDW; the overall structure and magnitude of the net transformation primarily reflects the freshwater fluxes.

To identify the role of sea ice, in Fig.~2b we decompose the freshwater-induced transformation into contributions from direct precipitation, brine rejection (ice freezing), and ice melting---these three sum to give the total freshwater-induced transformation. (The net melting exceeds the freezing because the melt flux also includes intercepted snow.)


 the freshwater-induced transformation into an atmospheric component and a sea-ice-redistribution component. The sea ice redistribution is responsible for all of the freshwater loss and resulting densification (15 Sv) in the Antarctic zone and contributes significantly to the freshwater gain and lightening (10 Sv) in the upper branch. This effect is comparable in magnitude the transformation by heat fluxes and, moreover, strongly modulates the spatial structure of water mass transformation. Without sea ice redistribution, dense Antarctic water would not be formed at all, and the transformation of upwelling CDW would look very different.
 
 We also show the net effect of sea ice freshwater redistribution by plotting the transformation rate due to the redistribution flux from Fig. 1; this is the difference between the actual transformation rate and the transformation that would occur if freshwater leaving the atmosphere were transferred directly to the ocean.

Finally, in Fig.~2c, we compute the lateral freshwater flux, in potential density space, due to sea ice transport and atmospheric transport. At high latitudes, the atmosphere moves moisture polewards (towards higher ocean density), and the divergence of this flux falls as precipitation. The ice clearly acts in the opposite sense, moving freshwater from low density to high. The magnitude of the sea ice transport is comparable to the atmospheric transport, indicating that the role of sea ice in the hydrological cycle of the Antarctic region is comparable to the atmosphere.

\begin{figure}
\begin{center}
\includegraphics[width=5in]{transformation.pdf}
\caption{Density transformation rates due to surface heat and freshwater fluxes, and their sum, the net surface transformation rate. The freshwater flux is further decomposed into an atmospheric component and a sea-ice redistribution component, which sum to give the total freshwater transformation rate.}
\label{fig:transformation}
\end{center}
\end{figure}

The implication of our findings is that sea ice freshwater redistribution must play a central role in the global ocean overturning system. Much interest and effort has focused on the {\em extent} of sea ice: its recent history\cite{some}, its expected trends with climate change\cite{relevant}, and its variability in paleoclimate\cite{FerrariEtAl2014}. For ocean circulation, however, the freshwater flux, rather than the ice extent, is the most relevant quantity; the two are related but not necessarily proportional. The sea ice extent is constrained by the water and air temperature, but the winds play the dominant role in driving ice motion\cite{HollandKwok2012,HollandEtAl2014}. Based on our findings, we except that wind-driven variability in sea ice transport can drive substantial variability in ocean overturning circulation. 

\begin{figure}
\begin{center}
\includegraphics[width=5in]{cartoon.pdf}
\caption{Schematic.}
\label{fig:cartoon}
\end{center}
\end{figure}

\section*{Methods}

\section*{Supplementary Material}

\begin{align}
&\od{}{t}FW_{atm} & = &- F_{A \to O} & -F_{A \to I}  & \ 		  & \\
&\od{}{t}FW_{ice}  & = & \		       & + F_{A \to I} & - F_{I \to O}& \\
&\od{}{t}FW_{oc}  & =  & F_{A \to O}  & \                  & + F_{I \to O}& 
\end{align}

It follows that
\begin{equation}
\od{}{t}FW_{ice} = -\od{}{t}FW_{atm} - \od{}{t}FW_{oc} 
\end{equation}


\section*{Another Section}

Sections can only be used in Articles.  Contributions should be
organized in the sequence: title, text, methods, references,
Supplementary Information line (if any), acknowledgements,
interest declaration, corresponding author line, tables, figure
legends.

Spelling must be British English (Oxford English Dictionary)

In addition, a cover letter needs to be written with the
following:
\begin{enumerate}
 \item A 100 word or less summary indicating on scientific grounds
why the paper should be considered for a wide-ranging journal like
\textsl{Nature} instead of a more narrowly focussed journal.
 \item A 100 word or less summary aimed at a non-scientific audience,
written at the level of a national newspaper.  It may be used for
\textsl{Nature}'s press release or other general publicity.
 \item The cover letter should state clearly what is included as the
submission, including number of figures, supporting manuscripts
and any Supplementary Information (specifying number of items and
format).
 \item The cover letter should also state the number of
words of text in the paper; the number of figures and parts of
figures (for example, 4 figures, comprising 16 separate panels in
total); a rough estimate of the desired final size of figures in
terms of number of pages; and a full current postal address,
telephone and fax numbers, and current e-mail address.
\end{enumerate}

See \textsl{Nature}'s website
(\texttt{http://www.nature.com/nature/submit/gta/index.html}) for
complete submission guidelines.

\begin{methods}
Put methods in here.  If you are going to subsection it, use
\verb|\subsection| commands.  Methods section should be less than
800 words and if it is less than 200 words, it can be incorporated
into the main text.

\subsection{Method subsection.}

Here is a description of a specific method used.  Note that the
subsection heading ends with a full stop (period) and that the
command is \verb|\subsection{}| not \verb|\subsection*{}|.

\end{methods}

%% Put the bibliography here, most people will use BiBTeX in
%% which case the environment below should be replaced with
%% the \bibliography{} command.

\bibliography{../../bibliography/references.bib}

%% Here is the endmatter stuff: Supplementary Info, etc.
%% Use \item's to separate, default label is "Acknowledgements"

\begin{addendum}
 \item Put acknowledgements here.
 \item[Competing Interests] The authors declare that they have no
competing financial interests.
 \item[Correspondence] Correspondence and requests for materials
should be addressed to A.B.C.~(email: myaddress@nowhere.edu).
\end{addendum}

%%
%% TABLES
%%
%% If there are any tables, put them here.
%%

\end{document}
